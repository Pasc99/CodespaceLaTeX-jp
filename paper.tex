%!TEX program = platex
%!TEX encoding = UTF-8 

\UseRawInputEncoding
\documentclass[12pt]{jreport}
\usepackage[top=25mm, bottom=35mm, left=25mm, right=25mm, footskip=10mm, headheight=0mm, headsep=0mm]{geometry}

%\def\bm#1{\mbox{\boldmath $#1$}}←bmパッケージ使えばいい
\usepackage{bm}
\usepackage[dvipdfmx]{graphicx}
\usepackage[dvipdfmx]{color}

\usepackage{here}
\usepackage{thesis}
%\usepackage{slashbox}
\usepackage{amsmath}
\usepackage{cases}
\usepackage{mathtools}
\usepackage[varg]{txfonts}
\usepackage{theorem}
\usepackage{url}
\usepackage{booktabs}
\usepackage{multirow}
\usepackage{enumitem}
\setenumerate{topsep=0zw,parsep=0zw,partopsep=0zw,itemsep=0zw,leftmargin=*,labelindent=1.25zw,labelsep=0.5zw}
\setenumerate[2]{labelindent=0zw,label=(\alph*),widest=a}
\setitemize{topsep=0zw,parsep=0zw,partopsep=0zw,itemsep=0zw,leftmargin=*,labelindent=1.25zw,labelsep=0.5zw}
\usepackage{cite}

% Algorithm environment
\newenvironment{algorithm}[1]{%
	\vskip 0.5\baselineskip%
	\hrule%
	\vskip 0.5\baselineskip%
	\centerline{\textbf{#1}}%
	\vskip 0.5\baselineskip%
	\hrule%
	\vskip 0.5\baselineskip%
	\setenumerate[1]{label=\arabic*.,ref=\arabic*,labelindent=1zw,listparindent=1zw,rightmargin=1zw}
	\setenumerate[2]{labelindent=0zw,label=(\alph*),widest=a}
	\begin{enumerate}%
}{%
	\end{enumerate}%
	\vskip 0.5\baselineskip%
	\hrule%
	\vskip 0.5\baselineskip%
	\leavevmode
}

% 必要ならここでパッケージを追加する

\usepackage{ascmac}
\usepackage{amsmath}

% MS明朝・MSゴシック・Times系フォントをPDFに埋め込むための設定
% \AtBeginDvi{\special{pdf:mapfile msembed.map}}
\AtBeginDvi{\special{pdf:mapfile map/IPAex_embed.map}}
\AtBeginDvi{\special{pdf:mapfile map/dlbase14.map}}

\theoremstyle{break}
\theorembodyfont{\normalfont}

\newtheorem{theo}{定理}[chapter]
\newtheorem{defi}[theo]{定義}
\newtheorem{lemm}[theo]{補題}
\newtheorem{col}[theo]{系}
\newtheorem{ass}[theo]{仮定}

% \topmargin 0cm
% \textheight 22.5cm
% \textwidth 16cm
% \oddsidemargin 5mm

\title{
{\Large 令和4年度\\[1mm]OO大学OOOO \\ 卒業論文} \\
\vspace{2cm}
\begin{minipage}[c]{16cm}
\begin{center}
{\LARGE Title}
\end{center}
\end{minipage}\\
\vspace{2cm}
{\Large 令和X年X月X日提出}
\vspace{2.5cm}
}


\author{
\Large XX分野 \vspace{5mm} \\
{\LARGE XX\ XX} \vspace{5mm} \\
指導教員:\ XX\ XX, XX\ XX
}


\date{}
\begin{document}
%%%%%%%%%%%%%%%%%%%%%%表紙%%%%%%%%%%%%%%%%%%%%%%%%%%%%
\maketitle
%%%%%%%%%%%%%%%%%%%表紙ここまで%%%%%%%%%%%%%%%%%%%%%%%%%

\renewcommand{\thepage}{\roman{page}}


%%%%%%%%%%%%%%%%%%%%%%概要%%%%%%%%%%%%%%%%%%%%%%%%%%%%
\newpage
%%%%%%%%%%%%%%%%%%%%%%概要%%%%%%%%%%%%%%%%%%%%%%%%%%%%
\vspace*{30pt}
\begin{center}
  {\Huge \textbf{概 要}}
\end{center}
\vspace{0.4cm}

%%%%%%%%%%%%%%%%%%%概要ここまで%%%%%%%%%%%%%%%%%%%%%%%%
%%%%%%%%%%%%%%%%%%%%%%Abstract%%%%%%%%%%%%%%%%%%%%%%%%%%%%
\newpage
\vspace{0.8cm}
%
\vspace*{4em}
\begin{center}
  %英語タイトル
  {\Large \textbf{0000}\\ }
  \vspace{2em}
  %名前
  \textbf{xxx} \\
  \vspace{0em}
\end{center}

\noindent
\textbf{Abstract : }
%This article is a template of the graduation thesis.
%The English abstract is written in this section.

\vspace{1em}
\noindent
\par
\textbf{Key words : }
%%%%%%%%%%%%%%%%%%%Abstractここまで%%%%%%%%%%%%%%%%%%%%%%%%%



%%%%%%%%%%%%%%%%%%%%%%目次%%%%%%%%%%%%%%%%%%%%%%%%%%%%
\newpage
\tableofcontents


%%%%%%%%%%%%%%%%%%%%%%第一章まえがき%%%%%%%%%%%%%%%%%%%%%%%%%%%%
\newpage
\renewcommand{\thepage}{\arabic{page}}
\setcounter{page}{1}

%%%%%%%%%%%%%%%%%%%%%%第一章まえがき%%%%%%%%%%%%%%%%%%%%%%%%%%%%
\chapter{まえがき}
%ここにまえがきを記述する。

%%%%%%%%%%%%%%%%%%%第一章ここまで%%%%%%%%%%%%%%%%%%%%%%%%%



%%%%%%%%%%%%%%%%%%%%%%第二章基礎的事項%%%%%%%%%%%%%%%%%%%%%%%%%%%%
\newpage
%%%%%%%%%%%%%%%%%%%%%%第二章基礎的事項%%%%%%%%%%%%%%%%%%%%%%%%%%%%
\chapter{基礎的事項}

%ここに基礎事項を記述する。
%セクションの記述の例を以下に示す。
%数式は下記の例のように中央揃え,align環境とする。

%%%%%%%%%%%%%%%%%%%第二章ここまで%%%%%%%%%%%%%%%%%%%%%%%%%



%%%%%%%%%%%%%%%%%%%%%%第三章提案手法%%%%%%%%%%%%%%%%%%%%%%%%%%%%
\newpage
%%%%%%%%%%%%%%%%%%%%%%第三章提案手法%%%%%%%%%%%%%%%%%%%%%%%%%%%%
\chapter{提案手法} \label{sec:proposal}

\section{XX法}













%%%%%%%%%%%%%%%%%%%第三章ここまで%%%%%%%%%%%%%%%%%%%%%%%%%


%%%%%%%%%%%%%%%%%%%%%%第四章計算機実験%%%%%%%%%%%%%%%%%%%%%%%%%%%%
\newpage
%%%%%%%%%%%%%%%%%%%%%%第四章実験%%%%%%%%%%%%%%%%%%%%%%%%%%%%
\chapter{実験}
\section{実験条件}
\clearpage
\section{考察}

%%%%%%%%%%%%%%%%%%%第四章ここまで%%%%%%%%%%%%%%%%%%%%%%%%%


%%%%%%%%%%%%%%%%%%%%%%第五章むすび%%%%%%%%%%%%%%%%%%%%%%%%%%%%
\newpage
%%%%%%%%%%%%%%%%%%%%%%第五章むすび%%%%%%%%%%%%%%%%%%%%%%%%%%%%
\chapter{むすび}

%%%%%%%%%%%%%%%%%%%第五章ここまで%%%%%%%%%%%%%%%%%%%%%%%%%


%%%%%%%%%%%%%%%%%%%%%%謝辞%%%%%%%%%%%%%%%%%%%%%%%%%%%%
\newpage
%%%%%%%%%%%%%%%%%%%%%%謝辞%%%%%%%%%%%%%%%%%%%%%%%%%%%%
\chapter*{謝辞}
\addcontentsline{toc}{chapter}{\protect\numberline{}{謝辞}}
%%%%%%%%%%%%%%%%%%%謝辞ここまで%%%%%%%%%%%%%%%%%%%%%%%%%



%%%%%%%%%%%%%%%%%%%%%%参考文献%%%%%%%%%%%%%%%%%%%%%%%%%%%%
\newpage
%%%%%%%%%%%%%%%%%%%%%%参考文献%%%%%%%%%%%%%%%%%%%%%%%%%%%%
\bibliographystyle{ref}
\bibliography{out}
%%%%%%%%%%%%%%%%%%%参考文献ここまで%%%%%%%%%%%%%%%%%%%%%%%%%



%%%%%%%%%%%%%%%%%%%%%%付録%%%%%%%%%%%%%%%%%%%%%%%%%%%%
%\newpage
%%%%%%%%%%%%%%%%%%%%%%%付録%%%%%%%%%%%%%%%%%%%%%%%%%%%%
\appendix

\chapter{実験結果補足}
ここに付録を記述する。
必要なければコメントアウトする。




%%%%%%%%%%%%%%%%%%%付録ここまで%%%%%%%%%%%%%%%%%%%%%%%%%


\end{document}
